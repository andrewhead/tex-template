% Packages that we may wish to include:
% Note that these should all be approved by TAPS:
% https://authors.acm.org/proceedings/production-information/accepted-latex-packages
% \usepackage{url}  % URLs
% \usepackage[normalem]{ulem}  % underlines and strikethroughs

% Colors
% \usepackage{xcolor}
\definecolor{andrewpurple}{HTML}{A53DFF}
\definecolor{andreworange}{HTML}{E07400}
\definecolor{darkgreen}{HTML}{009B55}

% Annotations
\newcommand\andrew[1]{\textcolor{andrewpurple}{#1}}
\newcommand\important[1]{\textcolor{darkgreen}{#1}}
\newcommand\unimportant[1]{\textcolor{gray}{\sout{#1}}}
\newcommand\move[1]{\textcolor{andreworange}{#1}}
\newcommand{\change}[1]{\textcolor{andrewpurple}{#1}}
\newenvironment{changes}
{\begingroup\color{andrewpurple}}
{\endgroup}

% Fonts
\def\computer#1{{\small\texttt{#1}}}
\AtBeginEnvironment{quote}{\itshape}

% Sectioning
\def\subparagraph#1{\textbf{#1.}}

% URL formatting
\def\UrlBigBreaks{\do\/\do-\do\#}

% Spacing helpers
\def\shortspace{\kern 0.1em}

% Common strings
\def\KaTeX{K\kern-.2em\raisebox{.2em}{\scriptsize A}\kern-.12em\TeX}

% Drawing boxes as isotypes in table
% Based on feedback from https://tex.stackexchange.com/questions/32597/vertically-centered-horizontal-rule-filling-the-rest-of-a-line
% and here https://texfaq.org/FAQ-rule#:~:text=The%20%5Crule%20command%20is%20used,as%20for%20characters%20with%20descenders).
% and here https://tex.stackexchange.com/questions/173042/is-it-possible-to-create-a-barchart-in-a-table
% and https://tex.stackexchange.com/questions/74353/what-commands-are-there-for-horizontal-spacing
% and https://moodle.org/mod/forum/discuss.php?d=431982
\definecolor{niceblue}{HTML}{8295ff}
\def\bigbox{\color{niceblue}\rule[.25ex]{1ex}{1ex} \hskip .1ex}
\def\smallbox{\hskip .25ex \color{gray}\rule[.5ex]{.5ex}{.5ex} \hskip .25ex \hskip .1ex}
\def\boxes#1#2{
\hskip .1ex % Add a bit of space, because this seems necessary for the boxes not to take up the whole line?
\newcount\boxnum
\boxnum=0
\loop
\ifnum \boxnum<#1 \bigbox \else \smallbox \fi

\advance \boxnum by 1
\ifnum \boxnum<#2
\repeat
}

% Inline figures
\newenvironment{inlinefigureenv}
{\setlength{\topsep}{2.5ex}\center}
{\endcenter}

\newcommand{\inlinefigure}[2][.5\textwidth]{%
\begin{inlinefigureenv}%
\includegraphics[width=#1]{#2}%
\vspace{-1.25ex}%
\end{inlinefigureenv}%
}
